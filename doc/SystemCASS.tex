%%%%%%%%%%
%
%$Log: SystemCASS.tex,v $
%Revision 1.3  2005/03/25 14:33:01  buchmann
%Typo :
%-  dependAncy -> dependEncy
%
%sc_initialize :
%-  Use a hash table to speed up elaboration step. (x40 faster)
%
%Tracing :
%-  check for modification BEFORE building bit string.
%-  use sprintf instead std string concatenation.
%
%Revision 1.2  2005/02/22 11:36:58  buchmann
%Generate correctly documentations.
%Add special flag to disable PAT tracing support.
%
%Revision 1.1  2005/01/20 09:15:12  buchmann
%add following functions to sc_uint classes :
%- operator []
%- range (left,right)
%
%support to port dependency declarations.
%print used precompiled options in verbose mode.
%use pedantic flag.
%add some rules to generate documentations.
%
%
%
%%%%%%%%%%%%%%%%%%%%%%%%%%%%%%%%
\documentclass{report}
\usepackage[latin1]{inputenc}
\usepackage[T1]{fontenc}
\usepackage{a4wide}
\usepackage{charter}
\usepackage{helvet}
\usepackage{graphicx}
\usepackage{amsmath}
\usepackage{moreverb}
\usepackage{longtable}
\usepackage{calc}
\usepackage{alltt}
\usepackage{makeidx}
\usepackage{xspace}
\renewcommand{\rmdefault}{phv}
\renewcommand{\sfdefault}{phv}
\renewcommand{\ttdefault}{pcr}
\newcommand{\SYSTEMCASS}{\textsc{SystemCASS}\xspace}
\newcommand{\SYSTEMC}{\textsc{SystemC}\xspace}
\newcommand{\SOCLIB}{\textsc{SoCLIB}\xspace}

\ifx\pdfoutput\undefined
\else
    \pdfpagewidth  21.0cm
    \pdfpageheight 29.7cm
\fi
\usepackage{hyperref}

\voffset   -1in
\hoffset   -1in
\headheight 0pt
\headsep    0pt
\topmargin     25mm
\oddsidemargin 29mm

\textwidth     150mm
\textheight    240mm
\flushbottom
\sloppy

\tabcolsep 4pt
\makeatletter
\renewcommand{\section}
   {\@startsection
   {section}%
   {1}%
   {0mm}%
   {-.5\baselineskip}%
   {0.5\baselineskip}%
   {\newpage\flushleft\normalfont\Large\scshape}}

\renewcommand{\subsection}
   {\@startsection
   {subsection}%
   {1}%
   {0mm}%
   {-.5\baselineskip}%
   {0.5\baselineskip}%
   {\flushleft\normalfont\large\scshape}}

\renewcommand{\@seccntformat}[1]{{\csname the#1\endcsname}.\hspace{0.25em}}
\makeatother
\renewcommand{\thesection}{\arabic{section}}

\renewcommand{\listinglabel}[1]{\rlap{\footnotesize\rmfamily\the#1}\hskip2em}

%%%%%%%%%%%%%%%%
%
%%%%%%%%%%%%%%%%%%%%%%%%%%%
\begin{document}
\thispagestyle{empty}
~\vfill
\begin{center}
{\LARGE SystemCASS\\ SystemC Cycle Accurate System Simulator\\}
\vfill
\vfill
%{\Large Documentation by Richard \textsc{Buchmann}}
\vfill

{\Large \today}

\end{center}
~\vfill

%%%%%%%%%%
%
%%%%%%%%%%%%%%%%%%%%%%%%%
\section{\SYSTEMCASS Overview}
\label{SystemCass-overview}

\SYSTEMCASS is a simulator that executes models described in \textbf{SystemC}.
Its goal is to provide cycle precise simulation of systems build with
hardware and software components, in order to evaluate performances
for \emph{i.e.} hardware/software partitioning, system validation, early
software development.

An example of system is present Figure~\ref{system}.

\begin{figure}[hbtp]\center\leavevmode
\includegraphics[width=.7\textwidth,angle=270]{SystemeType}
\caption{Typical embedded system build around VCI interfaces.}
\label{system}
\end{figure}

The simulator simulates a \textbf{SystemC} netlist of predefined and/or
user defined components.
The components used to validate the methodology\cite{iceec04-buch} are a part of \SOCLIB library.
Those models are VCI, bit and cycle accurate using the finite state machine with datapath modeling (Figure~\ref{Automate}).

\begin{figure}[hbtp]\center\leavevmode
\includegraphics[width=.6\linewidth]{Automate}
\caption{Finite State Machine Modeling}
\label{Automate}
\end{figure}

SystemCASS is a SystemC subset including~:
\begin{itemize}
\item
core class~: sc\_module, sc\_signal, sc\_in, sc\_out, sc\_inout...
\item
core functions~: sc\_start, sc\_stop, sc\_simulation\_time...
\item
basic data types~: sc\_i[u]nt, sc\_big[u]int...
\end{itemize}

But SystemCASS doesn't include standard channels, 
methodology-specific channels (master/slave library, verification library) 
and elementary channels (timer, mutex, semaphore, fifo, etc.).

This cycle accurate engine gives higher simulation speed 
more suitable for embedded software development,
and architectural exploration.

%%%%%%%%%%
%
%%%%%%%%%%%%%%%%%%%%%%%%%
\section{Installation}

Set \textbf{SYSTEMC} environment variable to the \textbf{SystemCASS} base directory.\\
Change the current directory to \textbf{\$(SYSTEMC)/src}.\\
Run the Makefile.\\

The distribution is as follow~:
\begin{itemize} \itemsep=-0.8ex
\item
\emph{src}~: 
Source files
\item
\emph{includes}~:
Header files
\item
\emph{lib-linux}~:
Debug and optimized libraries for linux distribution
\item
\emph{docs}~:
Documentation
\item
\emph{examples}~:

\begin{itemize} \itemsep=-0.8ex
\item
\emph{soclib\_date04}~:
An hardware timer raises some interruptions on 4 Mips R3000 periodically.
Whenever an interruption occurs, the application running on each CPU prints a "hello" message to its own output window.
\item
\emph{soclib\_spg}~:
An specific hardware configures a DMA. The DMA copies some data into the
a simple RAM component.
\end{itemize}

\end{itemize}

%%%%%%%%%%%%%%%%%%%%%%%
%
%%%%%%%%%%%%%%%%%%%%%%%%%%%%%%%%%%%%%%%%%%%%%%%%%%%%%%%%%�
\section{Simulator Execution}
\label{debbuger-usage}
To use SystemCASS simulator, you just need to include the SystemCASS headers
and link to SystemCASS library.

To debug, two ways are available~:
\begin{itemize}
\item
link to the debug library and use a debugger.
\item
generate a trace file by using dedicated functions and use a VCD viewer.
\end{itemize}

%%%%%%%%%%
%
%%%%%%%%%%%%%%%%%%%%%%%%%

\section{Simulator Compilation}

Some macro definitions are available 
to accelerate the simulation, 
to check the netlist, variable dependancies, the scheduling
to help on debugging
and so on.

You need to modify the Options.def file and rebuild the libraries.

\subsection{Scheduling}

The best performance you can reach using \SYSTEMCASS is obtained by using
static scheduling.

\SYSTEMCASS has two ways to compute the scheduling~:

\begin{itemize}
\item
from the static sensitivity list (default)~:
\SYSTEMC compatibility is good
\item
from the port dependency graph\cite{date04-mouchard} (USE\_PORT\_DEPENDENCY defined)~:
The component designer need to declare the port dependancies in the
constructor.
The signal dependency graph is complete and then it has no cycle.
At initialization step, the engine checks sensitivity list coherence.\\
\it{Syntax~: outPort(inPort);}\\
DUMP\_COMBINATIONAL\_LIST2DOT definition generates some DOT files to check
dependancies.
\end{itemize}

The generated scheduling is written in code-XXX.cc file.
KEEP\_GENERATED\_CODE definition allows to keep the file after simulation execution.\\

PRINT\_SCHEDULE definition prints the scheduling at execution time.\\
DUMP\_SCHEDULE\_STATS prints some miscellanous statistics.\\
NO\_STATIC\_SCHEDULE disables the static scheduling.

%%%%%%%%%%%%%%%
%\part{Creating new components for \SYSTEMCASS}
%%%%%%%%%%%%%%%

%\section{Component description}

%%%%%%%%%%%%%%%%%%%%%%%%%%%%%%
\bibliography{./SystemCASS}
\bibliographystyle{unsrt}
\end{document}

